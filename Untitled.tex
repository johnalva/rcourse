\documentclass[]{article}
\usepackage{lmodern}
\usepackage{amssymb,amsmath}
\usepackage{ifxetex,ifluatex}
\usepackage{fixltx2e} % provides \textsubscript
\ifnum 0\ifxetex 1\fi\ifluatex 1\fi=0 % if pdftex
  \usepackage[T1]{fontenc}
  \usepackage[utf8]{inputenc}
\else % if luatex or xelatex
  \ifxetex
    \usepackage{mathspec}
  \else
    \usepackage{fontspec}
  \fi
  \defaultfontfeatures{Ligatures=TeX,Scale=MatchLowercase}
\fi
% use upquote if available, for straight quotes in verbatim environments
\IfFileExists{upquote.sty}{\usepackage{upquote}}{}
% use microtype if available
\IfFileExists{microtype.sty}{%
\usepackage{microtype}
\UseMicrotypeSet[protrusion]{basicmath} % disable protrusion for tt fonts
}{}
\usepackage[margin=1in]{geometry}
\usepackage{hyperref}
\hypersetup{unicode=true,
            pdftitle={Curso R},
            pdfauthor={Johnny Alvarez},
            pdfborder={0 0 0},
            breaklinks=true}
\urlstyle{same}  % don't use monospace font for urls
\usepackage{color}
\usepackage{fancyvrb}
\newcommand{\VerbBar}{|}
\newcommand{\VERB}{\Verb[commandchars=\\\{\}]}
\DefineVerbatimEnvironment{Highlighting}{Verbatim}{commandchars=\\\{\}}
% Add ',fontsize=\small' for more characters per line
\usepackage{framed}
\definecolor{shadecolor}{RGB}{248,248,248}
\newenvironment{Shaded}{\begin{snugshade}}{\end{snugshade}}
\newcommand{\KeywordTok}[1]{\textcolor[rgb]{0.13,0.29,0.53}{\textbf{#1}}}
\newcommand{\DataTypeTok}[1]{\textcolor[rgb]{0.13,0.29,0.53}{#1}}
\newcommand{\DecValTok}[1]{\textcolor[rgb]{0.00,0.00,0.81}{#1}}
\newcommand{\BaseNTok}[1]{\textcolor[rgb]{0.00,0.00,0.81}{#1}}
\newcommand{\FloatTok}[1]{\textcolor[rgb]{0.00,0.00,0.81}{#1}}
\newcommand{\ConstantTok}[1]{\textcolor[rgb]{0.00,0.00,0.00}{#1}}
\newcommand{\CharTok}[1]{\textcolor[rgb]{0.31,0.60,0.02}{#1}}
\newcommand{\SpecialCharTok}[1]{\textcolor[rgb]{0.00,0.00,0.00}{#1}}
\newcommand{\StringTok}[1]{\textcolor[rgb]{0.31,0.60,0.02}{#1}}
\newcommand{\VerbatimStringTok}[1]{\textcolor[rgb]{0.31,0.60,0.02}{#1}}
\newcommand{\SpecialStringTok}[1]{\textcolor[rgb]{0.31,0.60,0.02}{#1}}
\newcommand{\ImportTok}[1]{#1}
\newcommand{\CommentTok}[1]{\textcolor[rgb]{0.56,0.35,0.01}{\textit{#1}}}
\newcommand{\DocumentationTok}[1]{\textcolor[rgb]{0.56,0.35,0.01}{\textbf{\textit{#1}}}}
\newcommand{\AnnotationTok}[1]{\textcolor[rgb]{0.56,0.35,0.01}{\textbf{\textit{#1}}}}
\newcommand{\CommentVarTok}[1]{\textcolor[rgb]{0.56,0.35,0.01}{\textbf{\textit{#1}}}}
\newcommand{\OtherTok}[1]{\textcolor[rgb]{0.56,0.35,0.01}{#1}}
\newcommand{\FunctionTok}[1]{\textcolor[rgb]{0.00,0.00,0.00}{#1}}
\newcommand{\VariableTok}[1]{\textcolor[rgb]{0.00,0.00,0.00}{#1}}
\newcommand{\ControlFlowTok}[1]{\textcolor[rgb]{0.13,0.29,0.53}{\textbf{#1}}}
\newcommand{\OperatorTok}[1]{\textcolor[rgb]{0.81,0.36,0.00}{\textbf{#1}}}
\newcommand{\BuiltInTok}[1]{#1}
\newcommand{\ExtensionTok}[1]{#1}
\newcommand{\PreprocessorTok}[1]{\textcolor[rgb]{0.56,0.35,0.01}{\textit{#1}}}
\newcommand{\AttributeTok}[1]{\textcolor[rgb]{0.77,0.63,0.00}{#1}}
\newcommand{\RegionMarkerTok}[1]{#1}
\newcommand{\InformationTok}[1]{\textcolor[rgb]{0.56,0.35,0.01}{\textbf{\textit{#1}}}}
\newcommand{\WarningTok}[1]{\textcolor[rgb]{0.56,0.35,0.01}{\textbf{\textit{#1}}}}
\newcommand{\AlertTok}[1]{\textcolor[rgb]{0.94,0.16,0.16}{#1}}
\newcommand{\ErrorTok}[1]{\textcolor[rgb]{0.64,0.00,0.00}{\textbf{#1}}}
\newcommand{\NormalTok}[1]{#1}
\usepackage{graphicx,grffile}
\makeatletter
\def\maxwidth{\ifdim\Gin@nat@width>\linewidth\linewidth\else\Gin@nat@width\fi}
\def\maxheight{\ifdim\Gin@nat@height>\textheight\textheight\else\Gin@nat@height\fi}
\makeatother
% Scale images if necessary, so that they will not overflow the page
% margins by default, and it is still possible to overwrite the defaults
% using explicit options in \includegraphics[width, height, ...]{}
\setkeys{Gin}{width=\maxwidth,height=\maxheight,keepaspectratio}
\IfFileExists{parskip.sty}{%
\usepackage{parskip}
}{% else
\setlength{\parindent}{0pt}
\setlength{\parskip}{6pt plus 2pt minus 1pt}
}
\setlength{\emergencystretch}{3em}  % prevent overfull lines
\providecommand{\tightlist}{%
  \setlength{\itemsep}{0pt}\setlength{\parskip}{0pt}}
\setcounter{secnumdepth}{0}
% Redefines (sub)paragraphs to behave more like sections
\ifx\paragraph\undefined\else
\let\oldparagraph\paragraph
\renewcommand{\paragraph}[1]{\oldparagraph{#1}\mbox{}}
\fi
\ifx\subparagraph\undefined\else
\let\oldsubparagraph\subparagraph
\renewcommand{\subparagraph}[1]{\oldsubparagraph{#1}\mbox{}}
\fi

%%% Use protect on footnotes to avoid problems with footnotes in titles
\let\rmarkdownfootnote\footnote%
\def\footnote{\protect\rmarkdownfootnote}

%%% Change title format to be more compact
\usepackage{titling}

% Create subtitle command for use in maketitle
\newcommand{\subtitle}[1]{
  \posttitle{
    \begin{center}\large#1\end{center}
    }
}

\setlength{\droptitle}{-2em}

  \title{Curso R}
    \pretitle{\vspace{\droptitle}\centering\huge}
  \posttitle{\par}
    \author{Johnny Alvarez}
    \preauthor{\centering\large\emph}
  \postauthor{\par}
      \predate{\centering\large\emph}
  \postdate{\par}
    \date{9/15/2018}


\begin{document}
\maketitle

\subsection{Introduccion}\label{introduccion}

\emph{R = Lenguaje de programacion estadistica} \emph{R Studio =
compañía que diseño la IDE para el desarrollo e interpretación de R}

\subsubsection{RStudio Desktop IDE:}\label{rstudio-desktop-ide}

Es la aplicación que se puede correr en los diferentes sistemas
operativos, por ejemplo Windows, Mac, Linux entre otros

\subsubsection{RStudio Server Open
Source:}\label{rstudio-server-open-source}

Este servidor corresponde a un servidor virtual el cual puede ser
instalado en ambientes de Lunix como Redhat, Fedora, Ubunto, etc.
Permite al usuario tener la experiencia de hacer deploys dentro de un
servidor, optimizar recursos y aprovechar mejor la memoria de la
computadora.

\subsubsection{RStudio Server Pro:}\label{rstudio-server-pro}

Al igual que el servidor open source el Server Pro permite a los
usuarios hacer deploy en un servidor virtual, sin emgargo esta es una
versión de paga, con características mas avanzadas en términos de
seguirdad, métricas de seguirdad, sesiones multiples, entre otros.

\subsubsection{Links de descarga:}\label{links-de-descarga}

Para descargar R: \url{https://cran.r-project.org} Para descargar
R-Studio: \url{https://www.rstudio.com/products/rstudio/download/}

\subsection{Consola:}\label{consola}

\begin{center}\rule{0.5\linewidth}{\linethickness}\end{center}

\begin{enumerate}
\def\labelenumi{\arabic{enumi}.}
\tightlist
\item
  Paneles
\end{enumerate}

\begin{itemize}
\tightlist
\item
  Consola = \emph{Interprete de R donde se puede escribir el código
  directo}
\item
  Texto = \emph{En esta sección se puede escribir segmentos de código
  (Métodos), poner comentarios, etc}
\item
  Ambiente = \emph{En esta pestaña, el desarrollador es capaz de
  visualizar los elementos que se van generando en la aplicación,
  revisar el historial, conecciones y documentos compartidos en GIT}
\item
  Archivos = \emph{En la pestaña de archivos el desarrollador puede ver
  cuales archivos se encunetran dentro de la carpeta de la aplicación,
  visualizar los gráficos (Plots), ver los paquetes disponible para
  instalar así como los instaldados, acceso a la sección de ayuda, y por
  último visualización de gráficos}
\end{itemize}

\textbf{Es importante tener en cuenta que los paneles son altamente
manipulables o sea que los puede modificar a gusto del desarrollador}

\subsection{Series}\label{series}

\subsubsection{\texorpdfstring{\textbf{Programming}}{Programming}}\label{programming}

\begin{itemize}
\tightlist
\item
  Writing code in RStudio
\item
  Debugging code in RStudio
\item
  Package writing in RStudio
\end{itemize}

\subsubsection{\texorpdfstring{\textbf{Managing
Change}}{Managing Change}}\label{managing-change}

\begin{itemize}
\tightlist
\item
  Projects in RStudio
\item
  Github and RStudio
\item
  Packrat and RStudio
\end{itemize}

\subsection{Writing code in RStudio}\label{writing-code-in-rstudio}

Algunis ejemplos:

\begin{Shaded}
\begin{Highlighting}[]
\KeywordTok{paste}\NormalTok{(}\StringTok{"Hello"}\NormalTok{, }\StringTok{"Word"}\NormalTok{)}
\end{Highlighting}
\end{Shaded}

\begin{verbatim}
## [1] "Hello Word"
\end{verbatim}

\begin{Shaded}
\begin{Highlighting}[]
\NormalTok{serie1 <-}\StringTok{ }\DecValTok{1}\OperatorTok{:}\DecValTok{10}
\NormalTok{serie2 <-}\StringTok{ }\DecValTok{11}\OperatorTok{:}\DecValTok{20}

\KeywordTok{print}\NormalTok{(serie1)}
\end{Highlighting}
\end{Shaded}

\begin{verbatim}
##  [1]  1  2  3  4  5  6  7  8  9 10
\end{verbatim}

\begin{Shaded}
\begin{Highlighting}[]
\NormalTok{serie2}
\end{Highlighting}
\end{Shaded}

\begin{verbatim}
##  [1] 11 12 13 14 15 16 17 18 19 20
\end{verbatim}

\emph{En IDE:} View(iris)

\begin{itemize}
\tightlist
\item
  Visualización
\item
  Función de filtrado
\end{itemize}


\end{document}
